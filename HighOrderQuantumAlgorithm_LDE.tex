\documentclass{article}
\PassOptionsToPackage{quiet}{fontspec}
\usepackage[UTF8]{ctex}
\usepackage{amsmath,amssymb,amsthm}
\usepackage{braket}
\usepackage{enumitem}
\usepackage{graphicx}
\usepackage{float} 
\usepackage{geometry}
\usepackage{enumitem}
\theoremstyle{definition}
\newtheorem{definition}{定义}
\newtheorem{theorem}{定理}
\newtheorem{lemma}{引理}
\geometry{top=2cm, bottom=2cm, textwidth=16cm}
\title{求解线性微分方程的高阶量子算法}
\author{王心锐}
\begin{document}
\maketitle

\section{线性系统方法}

为解决复杂度随演化时间 Δt 呈指数增长和无法求解非齐次线性微分方程的问题,我们提出一种基于HHL算法的方法。其关键在于引入“费曼时钟(Feynman's clock)”——通过一个额外的寄存器对时间进行编码。随后,我们仅用一个量子态就能对微分方程在所有时刻的解进行编码。也就是说,我们希望得到一个与之成比例的最终量子态:
\[
|\psi\rangle := \sum_{j=0}^{N_{t}} |t_{j}\rangle |x_{j}\rangle.
\]

其中,:=是定义符号,\(N_t\)为时间步数,\(t_j = t_0 + jh\)(\(h\)是微分方程离散化后的时间步长,\(t_0\)是微分方程求解的初始时刻),\(x_j\)是时刻\(t_j\)处向量\(x\)的近似值,\(\Delta t\)是求解微分方程的总时间跨度。本文中,我们用下标\(j\)表示向量的索引,用上标表示向量的分量。

一旦构建出这个量子态,通过测量编码时间的寄存器并得到最终时刻\(t_0+\Delta t\),就能近似得到编码该时刻解的量子态。不过,仅采用这种方法时,测得最终时刻的概率很低,仅为\(\frac{1}{N_t + 1}\)。为显著提高成功概率,我们可以在\(t_0+\Delta t\)之后增加一段“解保持恒定”的时间区间:令向量\(x\)在\(t_0+\Delta t\)到\(t_0 + 2\Delta t\)时间段内保持不变,此时时间步数满足\(N_t = \frac{2\Delta t}{h}\)。这样一来,只要测量到的时间落在该区间内,就能得到对应解的量子态。通过这种方法,成功概率可显著提高,且不会改变\(N_t\)的复杂度缩放趋势。

数值求解微分方程的最简单方法是欧拉法(Euler method),其将微分方程离散化为:
\[
\frac{x_{j+1}-x_{j}}{h} = A(t_{j}) x_{j} + b(t_{j}). 
\]

对\(t > t_0+\Delta t\)的时间段,我们令\(x_{j+1}=x_j\),以确保解的恒定。将欧拉法编码为线性方程组的形式十分直接,即:
\[
\mathcal{A} \vec{x} = \vec{b}. 
\]

其中,\(\vec{x}\)是由各时刻向量\(x_j\)构成的块向量,\(\vec{b}\)是由非齐次项\(b\)和初始条件\(x_{in}\)构成的块向量,\(\mathcal{A}\)是描述离散化微分方程的块矩阵。当\(A\)和\(b\)不随时间变化时,可给出如下简单示例:
\[
\underbrace{\begin{pmatrix}
\mathbf{I} & 0 & 0 & 0 & 0 \\
-(\mathbf{I} + Ah) & \mathbf{I} & 0 & 0 & 0 \\
0 & -(\mathbf{I} + Ah) & \mathbf{I} & 0 & 0 \\
0 & 0 & -\mathbf{I} & \mathbf{I} & 0 \\
0 & 0 & 0 & -\mathbf{I} & \mathbf{I}
\end{pmatrix}}_{\mathcal{A}} 
\underbrace{\begin{pmatrix}
x_0 \\
x_1 \\
x_2 \\
x_3 \\
x_4
\end{pmatrix}}_{\vec{x}} 
= 
\underbrace{\begin{pmatrix}
x_{in} \\
bh \\
bh \\
0 \\
0
\end{pmatrix}}_{\vec{b}}. 
\]

矩阵\(\mathcal{A}\)的每个元素都是与矩阵\(A\)同维度的块,向量\(\vec{x}\)和\(\vec{b}\)的每个元素都是与向量\(x\)同维度的块。其中:
\begin{itemize}
    \item 第一行用于设定初始值,即\(x_0 = x_{in}\);
    \item 后续几行对应欧拉法离散化的微分方程,即\(x_{j+1} - (x_j + A x_j h) = bh\);
    \item 最后几行对应“解保持恒定”的约束条件,即\(x_{j+1} - x_j = 0\)。
\end{itemize}

欧拉法的单步误差缩放阶为\(O(h^2)\),因此总模拟误差可表示为\(O(N_t h^2) = O\left(\frac{\Delta t^2}{N_t}\right)\)。若要求总误差不超过\(\epsilon\)(此处\(\epsilon\)为微分方程求解的允许误差,与线性方程组求解的允许误差\(\epsilon_L\)相区分),则需选择时间步数\(N_t = O\left(\frac{\Delta t^2}{\epsilon}\right)\)。要确定通过HHL算法求解该线性方程组的复杂度,需先估算其条件数。

对条件数的粗略估算可通过分析“给定\(\vec{b}\)时\(\vec{x}\)的最大可能规模”实现:直观上,\(\vec{b}\)的每个分量可视为一个“激励源”,该激励会导致后续时刻\(\vec{x}\)的数值增大;由于共有\(O(N_t)\)个时刻受此激励影响,因此\(\vec{x}\)的规模约为\(\vec{b}\)的\(O(N_t)\)倍。不难发现,矩阵\(\mathcal{A}\)的算子范数\(\|\mathcal{A}\| \approx 1\),因此其条件数\(\kappa(\mathcal{A}) = O(N_t)\)。这使得算法复杂度的缩放阶至少为\(O(\Delta t^4)\)(后续章节将对此进行严格证明)。

条件数的定义是: 
\[
\kappa(\mathcal{A}) = \| \mathcal{A} \| \cdot \| \mathcal{A}^{-1} \|
\]

其中\(\| \mathcal{A}^{-1} \|\)是逆矩阵\(\mathcal{A}^{-1}\)的算子范数,衡量逆矩阵对向量的最大放大能力。

结合式子的物理意义:离散化矩阵\(\mathcal{A}\)对应的线性系统\(\mathcal{A}\vec{x} = \vec{b}\)中,\(\vec{b}\)是输入(如初始条件\(x_{\text{in}}\)、激励\(bh\)),\(\vec{x}\)是输出(各时间步的状态\(x_0, x_1, \dots, x_{N_t}\))。由于状态\(\vec{x}\)是通过\(N_t\)个时间步累积得到的(每一步的状态依赖前一步),逆矩阵\(\mathcal{A}^{-1}\)的作用本质是“从输入\(\vec{b}\)还原出累积的状态\(\vec{x}\)”,其放大能力会随时间步数\(N_t\)增长——即
\[
\| \mathcal{A}^{-1} \| \approx O(N_t)
\](与时间步数成正比)。

再结合\(\| \mathcal{A} \| \approx 1\),代入条件数定义可得: 
\[
\kappa(\mathcal{A}) = \| \mathcal{A} \| \cdot \| \mathcal{A}^{-1} \| \approx 1 \cdot O(N_t) = O(N_t)
\]

相较于哈密顿模拟中\(O(\Delta t)\)(或接近线性)的时间缩放阶,上述\(\Delta t\)的缩放效果并不理想。与哈密顿模拟类似,我们可通过高阶方法改善缩放性能——对于微分方程,适用的高阶方法是线性多步法(Linear Multistep Methods)。第四部分将介绍多步法的基本原理,第五部分将说明如何对条件数进行界估计,第六部分则会阐述HHL算法的具体应用及最终复杂度。


\section{多步法}

线性多步法的通用形式如下:
\[
\sum_{\ell=0}^{k} \alpha_{\ell} x_{j+\ell} = h \sum_{\ell=0}^{k} \beta_{\ell} \left[A(t_{j+\ell}) x_{j+\ell} + b(t_{j+\ell})\right]. 
\]

选择合适的多步法可使误差在步长\(h\)上达到更高阶,但该方法存在稳定性问题。为保证误差有界,矩阵\(A(t_j)\)的特征值需满足“实部非正”——否则误差会呈指数增长。若\(A(t_j)\)存在实部为正的特征值,可通过减去单位矩阵的若干倍并对解进行重新缩放来处理。即便如此,当微分方程的精确解有界时,差分方程的数值解仍可能无界。

为分析稳定性,定义生成多项式:
\[
\rho(\zeta) = \sum_{j=0}^{k} \alpha_{j} \zeta^{j}, \quad \sigma(\zeta) = \sum_{j=0}^{k} \beta_{j} \zeta^{j}.
\]

稳定性可通过以下方程的根进行分析:
\[
\rho(\zeta) - \mu \sigma(\zeta) = 0. 
\]

定义集合\(S\):
\[
S := \left\{ \mu \in \mathbb{C} \mid \begin{array}{l} \text{方程\(
\rho(\zeta) - \mu \sigma(\zeta) = 0 
\)的所有根}\ \zeta_j(\mu) \\ \ \text{满足}\ |\zeta_j(\mu)| \leq 1 \text{重根满足}\ |\zeta_j(\mu)| < 1 \end{array} \right\}. 
\]

集合\(S\)被称为多步法的稳定域(Stability Domain)。此外,若多项式\(\sigma(\zeta)\)的所有根满足\(|\zeta| \leq 1\),且重根满足\(|\zeta| < 1\),则称该多步法在“无穷远处稳定(stable at infinity)”。

若线性多步法的局部误差缩放阶为\(O(h^{p+1})\),则称该方法为\(p\)阶方法。这意味着:若将该方法应用于测试方程\(\dot{x} = t^q\)(\(0 \leq q \leq p\))且初始值精确,则该方法可无误差地对该方程进行积分。线性多步法为\(p\)阶的充要条件是:
\[
\rho(e^h) - h \sigma(e^h) = O(h^{p+1}). 
\]

线性多步法的一个重要性质是“A-稳定性(A-stable)”。

\begin{definition}[A-稳定性]
若线性多步法的稳定域\(S\)包含整个左半复平面(\(\mathbb{C}^- = \{\mu \in \mathbb{C} \mid \text{Re}(\mu) \leq 0\}\)),即:
\[
\text{若}\ \text{Re}(\lambda) \leq 0,\ \text{则方程}\ \dot{x} = \lambda x\ \text{的数值解有界}
\]
\end{definition}

该定义表明:若微分方程的精确解有界,则多步法给出的近似解也有界。对于标量微分方程,只要\(\lambda\)落在左半复平面内,多步法的数值解就有界。虽然欧拉法具备A-稳定性,但无法构造任意阶数的A-稳定多步法——这就是“达尔奎斯特第二障碍(Second Dahlquist Barrier)”:A-稳定线性多步法的阶数\(p \leq 2\)。由于本文需考虑高阶多步法,因此放宽稳定性要求,采用“A (\(\alpha\))-稳定性(A (\(\alpha \))-stable)”。

\begin{definition}[A (\(\alpha \))-稳定性]
对于\(0 < \alpha < \frac{\pi}{2}\),若线性多步法的稳定域\(S\)包含“左半复平面内的\(\alpha \)-楔形区域”\(S_\alpha\),即:
\[
S \supset S_\alpha = \{\mu \in \mathbb{C} \mid |\arg(-\mu)| < \alpha,\ \mu \neq 0\}, 
\]
则称该线性多步法是A (\(\alpha \))-稳定的。
\end{definition}

该定义表明:
\begin{itemize}
    \item 对于标量微分方程,只要特征值\(\lambda\)落在左半复平面内的\(\alpha\)-楔形区域(即\(|\arg(-\lambda)| \leq \alpha\)),多步法的数值解就有界;
    \item 对于向量微分方程,只需系数矩阵\(A\)的所有特征值满足上述\(\alpha\)-楔形区域条件即可。
\end{itemize}

已知结论为:对于任意\(\alpha < \frac{\pi}{2}\)和正整数\(k\),存在“\(k\)步、\(p=k\)阶”的\(A(\alpha)\)-稳定线性多步法。


微分方程数值解的总误差缩放阶为 \(O\left(N_t {\Delta t}^{p+1}\right)\)。为得到严格的误
差界,本文限定系数矩阵 \(A\) 和非齐次项 \(b\) 与时间无关(即线性定常微分方程)。



\begin{theorem}
设线性多步法为\(p\)阶方法,且满足A (\(\alpha \))-稳定和无穷远处稳定。若矩阵\(A\)可对角化(即存在矩阵\(V\),使得\(V^{-1} A V = D = \text{diag}(\lambda_1, \dots, \lambda_{N_x})\)),且其特征值满足:
\[
|\arg(-\lambda_i)| \leq \alpha\quad (i=1,2,\dots,N_x), 
\]
则存在仅依赖于该多步法的常数\(M\),使得对所有\(h > 0\),全局误差满足:
\[
\|x(t_m) - x_m\| \leq M \kappa_V \left( \max_{0 \leq j < k} \|x(t_j) - x_j\| + h^p \int_{t_0}^{t_m} \|x^{(p+1)}(\xi)\| d\xi \right),
\]
其中\(\kappa_V = \|V\| \cdot \|V^{-1}\|\)是矩阵\(V\)的条件数,上标\((p+1)\)表示向量的\((p+1)\)阶导数。
\end{theorem}

利用上述定理,可得到关于误差缩放阶的引理:

\begin{lemma}
设线性多步法为\(p\)阶方法,且满足A (\(\alpha \))-稳定和无穷远处稳定。若矩阵\(A\)可对角化(即存在矩阵\(V\),使得\(V^{-1} A V = D = \text{diag}(\lambda_1, \dots, \lambda_{N_x})\)),其特征值满足:
\[
|\arg(-\lambda_i)| \leq \alpha\quad (i=1,2,\dots,N_x),
\]
且非齐次项\(b\)为常数,则全局误差满足:
\[
\|x(t_m) - x_m\| = O\left( \kappa_V^2 \left( \|x_{init}\| + \frac{\|b\|}{\|A\|} 
\right) \left[ \kappa_V {h\|A\|}^2 + m {h\|A\|}^{p+1} \right] \right), 
\]
其中\(\kappa_V = \|V\| \cdot \|V^{-1}\|\)是矩阵\(V\)的条件数。
\end{lemma}

\begin{proof}
线性多步法需通过“启动方法(starting method)”获取\(0 < j < k\)时\(x_j\)的值,式中\(\max_{0 \leq j < k} \|x(t_j) - x_j\|\)项即源于启动方法的误差。此处可直接采用欧拉法作为启动方法,其误差缩放阶为\(O(h^2)\);虽也可采用更高阶的启动方法,但目前缺乏便于应用的严格误差结果。

对欧拉法的误差分析可直接利用定理1:欧拉法的步数\(k=1\)、阶数\(p=1\),且初始点误差为0(\(x(t_0) = x_0\)),因此定理1给出:
\[
\|x(t_j) - x_j\| \leq M_E \kappa_V h \int_{t_0}^{t_j} \|x^{(2)}(\xi)\| d\xi, 
\]
其中\(M_E\)是欧拉法对应的常数。将该式应用于\(0 \leq j < k\),可得:
\[
\max_{0 \leq j < k} \|x(t_j) - x_j\| \leq M_E \kappa_V h^2 (k-1) \max_{\xi \in [t_0, t_0 + (k-1)h]} \|x^{(2)}(\xi)\|.
\]

分析上述结果时,需对\(\|x^{(p+1)}(\xi)\|\)和\(\|x^{(2)}(\xi)\|\)进行上界估计,这两个量通常依赖于\(b(t)\)及其时间相关性。当\(b\)为常数时,微分方程的精确解为:
\[
x(t) = e^{A(t - t_0)} \left( x_{init} + A^{-1} b \right) - A^{-1} b. 
\]
其\(\ell\)阶导数为:
\[
x^{(\ell)}(t) = e^{A(t - t_0)} \left( A^\ell x_{init} + A^{\ell-1} b \right), 
\]
因此导数的范数满足:
\[
\begin{aligned}
\|x^{(\ell)}(t)\| &= \| V e^{D(t - t_0)} V^{-1} \left( A^\ell x_{init} + A^{\ell-1} b \right) \| \\
&\leq \kappa_V \left( \|A\|^\ell \|x_{init}\| + \|A\|^{\ell-1} \|b\| \right)
\end{aligned}
\]
上式第一行利用了矩阵\(A\)的对角化性质,第二行利用了特征值的约束条件\(|\arg(-\lambda_i)| \leq \alpha\)。

将上述结果代入定理1,可得误差界:
\[
\begin{aligned}
\|x(t_m) - x_m\| 
&\leq M \kappa_V \left( 
  \max_{0 \leq j < k} \|x(t_j) - x_j\| + 
  h^p \int_{t_0}^{t_m} \|x^{(p+1)}(\xi)\| d\xi 
\right) \\
&\leq M \kappa_V \left[ 
  M_E \kappa_V^2 h^2 k \left( \|x_{\text{init}}\| + \frac{\|b\|}{\|A\|} \right) \|A\|^2 + \right. \\
&\qquad \left. h^p (t_m - t_0) \kappa_V \left( \|x_{\text{init}}\| + \frac{\|b\|}{\|A\|} \right) \|A\|^{p+1} 
\right] \\
&= O\left( 
  \kappa_V^2 \left( \|x_{\text{init}}\| + \frac{\|b\|}{\|A\|} \right) 
  \left[ \kappa_V {h\|A\|}^2 + m {h\|A\|}^{p+1} \right] 
\right).
\end{aligned}
\]
\end{proof}

该结果表明:若忽略对多数参量的依赖及启动方法的误差,总误差随总时间\(\Delta t\)的缩放阶为\(O\left( {\|A\| h}^p \|A\| \Delta t \right)\)。若要求总误差不超过\(\epsilon\),则需选择时间步数:
\[
N_t = O\left( \frac{{\|A\| \Delta t}^{1 + \frac{1}{p}}}{\epsilon^{\frac{1}{p}}} \right).
\]
即所需时间步数随总时间\(\Delta t\)的增长接近线性。

接下来考虑如何将多步法编码为线性方程组\(\mathcal{A}\vec{x} = \vec{b}\)。前文已介绍欧拉法的编码方式,更一般地,我们采用“欧拉法启动→高阶多步法迭代→恒定解约束”的流程:对于\(N_t \geq 2k\),矩阵\(\mathcal{A}\)的块元素定义为:
\[
\begin{cases}
\mathcal{A}_{j,j} = \mathbf{I}, & 0 \leq j < k ,\ \frac{N_t}{2} < j \leq N_t, \\
\mathcal{A}_{j,j-1} = -(\mathbf{I}+ Ah), & 1 \leq j < k, \\
\mathcal{A}_{j,j - k + \ell} = \alpha_\ell \mathbf{I} - \beta_\ell Ah, & k \leq j \leq \frac{N_t}{2},\ 0 \leq \ell \leq k, \\
\mathcal{A}_{j,j-1} = -\mathbf{I}, & \frac{N_t}{2} < j \leq N_t.
\end{cases}
\]
采用该矩阵时需满足\(N_t \geq 2k\),否则缺乏足够的时间步启动线性多步法。向量\(\vec{b}\)的块元素定义为:
\[
\begin{cases}
b_0 = x_{in}, \\
b_j = bh, & 1 \leq j < k, \\
b_j = \sum_{\ell=0}^k \beta_\ell bh, & k \leq j \leq \frac{N_t}{2}, \\
b_j = 0, & \frac{N_t}{2} < j \leq N_t.
\end{cases} 
\]

本文要求矩阵\(A\)、向量\(b\)和初始向量\(x_{in}\)均为稀疏矩阵/向量,即任意行或列的非零元素个数不超过\(s\)。假设所用“预言机(oracle)”的形式与文献[8]一致:
\begin{itemize}[label={-}]
    \item 矩阵\(A\)的预言机是一个幺正算子,作用形式为:
    \[
    O_A |j,\ell\rangle |z\rangle = |j,\ell\rangle |z \oplus A^{[j,\ell]}\rangle, 
    \]
    其中\(\oplus\)表示模加法,\(A^{[j,\ell]}\)是矩阵\(A\)第\(j\)行第\(\ell\)列元素的二进制表示(上标表示块内索引)。
    \item 稀疏性预言机用于定位非零元素:设函数\(f(j,\ell)\)表示“第\(j\)列第\(\ell\)个非零元素所在的行索引”,则该预言机为幺正算子,作用形式为:
    \[
    O_F |j,\ell\rangle = |j, f(j,\ell)\rangle. 
    \]
\end{itemize}

由于矩阵\(A\)非厄米,还需类似的预言机定位“给定行中的非零元素”;同时,还需用于获取\(b\)和\(x_{in}\)非零元素“位置与数值”的预言机。这些预言机可确保高效制备与\(\vec{b}\)对应的初始量子态;此外,也可假设\(b\)和\(x_{in}\)满足文献[24]中“高效制备流程”的条件。

利用文献[12]中的算法求解上述线性方程组,其复杂度为\(\tilde{O}\left( \log(N) s^4 \kappa^2 / \epsilon_L \right)\),其中\(\kappa\)是矩阵\(\mathcal{A}\)的条件数,\(\epsilon_L\)是线性方程组求解的允许误差(需注意:文献[12]中\(s\)的指数误写为2,实际应为4)。需强调的是,\(\epsilon_L\)是线性方程组求解的允许误差,与微分方程求解的允许误差\(\epsilon\)是两个不同的量。

\end{document}